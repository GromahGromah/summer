\documentclass[a4paper, 11pt, nofonts, nocap, fancyhdr]{ctexart}

\usepackage[margin=60pt]{geometry}

\setCJKmainfont[BoldFont={方正黑体_GBK}, ItalicFont={方正楷体_GBK}]{方正书宋_GBK}
\setCJKsansfont{方正黑体_GBK}
\setCJKmonofont{方正仿宋_GBK}

\CTEXoptions[today=small]

\pagestyle{plain}

\renewcommand{\thesubsubsection}{Problem \Alph{subsubsection}.}
\newcommand{\problem}[1]{\subsubsection{#1}}

\title{Fudan ACM-ICPC Summer Training Camp 2015}
\author{Team 6 汤定一/马天翼/金杰}
\date{2015年8月20日}

\begin{document}

\maketitle

\section{概况}

本场训练,我们队伍在比赛中完成了5道题目,比赛后完成了4道题目,共完成9道题目。

\section{训练过程}

开始大说这套题是水题,然后我们就开始去找签到题,然后并没有找到可以秒的题。jj发现05是一道多重背包,就上去写了,1A。tdy觉得06可做,写完wa了一发,然后A了。mty一直在推02的公式,但是没有推出很强的结论。然后jj写11,但是一直wa了8发,tdy上去写09,3A。mty写02,TLE,jj换方法重写,还是T。接着tdy写01。01也一直wa,tdy和jj决定让tdy重写11,在重写过程中发现了jj的错误,jj修改后过掉。然后mty和jj帮tdy找错误,找出错误后修改过掉,8A。最后mty和jj想找02的规律,但是由于时间过短找不到,比赛结束。

\section{解题报告}

\problem{CRB and Apple}%A

\begin{description}
\item[负责] 汤定一
\item[情况] 比赛中通过 - 279min(8Y)
\item[题意]
给定n个苹果,每个苹果有高度值、美味值。两个人一起吃苹果,只能从高往低吃,美味值从小往大吃,问最多能吃多少个。
\item[题解]
把苹果按高度为第一关键字,美味值为第二关键字排序。建费用流图,每个苹果拆点,美味值小的往大的连边,跑两次费用流即可。注意建边的时候有些边是不用建的,比如高度a>b>c,美味值a<b<c,只用建a到b的边和b到c的边,不用建a到c的边。
\end{description}

\problem{CRB And Candies}

\begin{description}
\item[负责] 马天翼、金杰
\item[情况] 比赛后通过
\item[题意]
求C(n,0)、C(n,1)、……、C(n,n)的LCM。
\item[题解]
找规律可得。。。。。
\end{description}

\problem{CRB and Farm}

\begin{description}
\item[负责] 金杰
\item[情况] 比赛后通过
\item[题意]
凸包内有k个点,要求在凸包的顶点中选取最多2k个点,使得这些点形成的新凸包能完全包含这k个点。
\item[题解]
先对k个点求凸包,注意到一旦大凸包不能完全包含小凸包,一定是有一个点出去了,所以从小凸包内部任取一点,向小凸包上每个顶点连出一条射线,交于大凸包的一条边,选取该边的两个顶点,就一定能把所以小凸包顶点包进去。又因为小凸包顶点最多k个,我最多选取2k个,刚好满足条件。
\end{description}

\problem{CRB and Graph}%D

\begin{description}
\item[负责] 汤定一
\item[情况] 比赛后通过
\item[题意]
给定无向图,对每条割边,输出u、v,若删掉割边e,点u、v不连通,u<v,u尽量大时v尽量小。
\item[题解]
tarjan求割边,缩点。以n所在的块为根,每条割边的答案即为其子树中最大的结点值u,u+1,因为u+1肯定不在子树中,此时答案最优。
\end{description}

\problem{CRB and His Birthday}

\begin{description}
\item[负责] 金杰
\item[情况] 比赛中通过 - 33min(1Y)
\item[题意]
给n种物品的体积,给背包容量,第i种物品取x(x>0)个的价值是ai*x+bi,问最大价值。
\item[题解]
把每种物品拆成一个限制数量1的价值为ai+bi的物品和无限制数量的价值为ai的物品,然后用多重背包队列优化即可。
\end{description}

\problem{CRB and Puzzle}%F

\begin{description}
\item[负责] 汤定一
\item[情况] 比赛中通过 - 81min(1Y)
\item[题意]
给定n个节点以及每个结点后可接的结点,问长度不超过m的串的不重复数量。
\item[题解]
矩阵,前n维记录以i结点结尾的串的不重复个数。加一维记录答案。
\end{description}

\problem{CRB and Queries}

\begin{description}
\item[情况] 尚未通过
\end{description}

\problem{CRB and Roads}%H

\begin{description}
\item[负责] 汤定一
\item[情况] 比赛后通过
\item[题意]
给定有向无环图,求非必须边个数,对一条边(u,v)若u能通过其他路径到v,称(u,v)为非必须边。
\item[题解]
拓扑序,从上往下,用bitset记录能到达i的结点,每个点的反边按拓扑逆序排序后边查边做即可。
\end{description}

\problem{CRB and String}%I

\begin{description}
\item[负责] 汤定一
\item[情况] 比赛中通过 - 131min(3Y)
\item[题意]
给定a串b串,求能否把a串变成b串。可以在a串其任意字符u后面加一个字符v(u!=v)。
\item[题解]
贪心。从前往后,找到a串中的每个字符i依次在b串中出现的位置,若i在b串中连续出现,则连续出现的最后一个,此时若a串也连续出现字符i把这些都放在b串连续出现的位置。注意特殊判断第一个字符。
\end{description}

\problem{CRB and Substrings}

\begin{description}
\item[情况] 尚未通过
\end{description}

\problem{CRB and Tree}

\begin{description}
\item[负责] 金杰
\item[情况] 比赛中通过 - 247min(9Y)
\item[题意]
树上的边有边权,给定m,问有多少对u,v,从u到v的路径上所有权值xor和为m。
\item[题解]
先以1为root跑一遍从root到i的xor和为a[i],只要统计有多少j使得a[j]=a[i] xor m,因为lca上边的那些都两次xor掉了。注意m=0的情况有些不同,要先统计非u,u的情况除以2,再加上u,u的情况不除以2。
\end{description}

\section{总结}

决定重写的时机不太好掌握,因为没有一个显而易见的肯定正确的策略在那里。\\
今天已经是第三次找规律题没有找规律而做不出来了,如果可以打表找规律的题一定要试试。

\end{document}
