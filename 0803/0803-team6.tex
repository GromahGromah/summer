\documentclass[a4paper, 11pt, nofonts, nocap, fancyhdr]{ctexart}

\usepackage[margin=60pt]{geometry}

\setCJKmainfont[BoldFont={方正黑体_GBK}, ItalicFont={方正楷体_GBK}]{方正书宋_GBK}
\setCJKsansfont{方正黑体_GBK}
\setCJKmonofont{方正仿宋_GBK}

\CTEXoptions[today=small]

\pagestyle{plain}

\renewcommand{\thesubsubsection}{Problem \Alph{subsubsection}.}
\newcommand{\problem}[1]{\subsubsection{#1}}

\title{Fudan ACM-ICPC Summer Training Camp 2015}
\author{Team 6 汤定一/马天翼/金杰}
\date{2015年8月3日}

\begin{document}

\maketitle

\section{概况}

本场训练,我们队伍在比赛中完成了5道题目,比赛后完成了2道题目,共完成7道题目。

\section{训练过程}

我是过程。

\section{解题报告}

\problem{我是A题的标题}

\begin{description}
\item[负责] 负责一、负责二
\item[情况] 比赛中通过 - 233min(1Y)
\item[题意]
我是题意。\\这是换行
回车不换行。
\item[题解]
我是题解。
\end{description}

\problem{我是B题的标题}

\begin{description}
\item[负责] 负责一、负责二
\item[情况] 比赛中通过 - 233min(1Y)
\item[题意]
我是题意。
\item[题解]
我是题解。
\end{description}

\problem{我是C题的标题}

\begin{description}
\item[负责] 负责一、负责二
\item[情况] 比赛后通过
\item[题意]
我是题意。
\item[题解]
我是题解。
\end{description}

\problem{Damage Assessment}

\begin{description}
\item[负责] 马天翼
\item[情况] 尚未通过
\item[题意]
给出一个胶囊装的油桶,告诉你一系列数据,求油桶内的油的体积。
\item[题解]
将油桶放平,建系后,辛普森即可。
\end{description}

\problem{Epic Win!}

\begin{description}
\item[负责] 金杰、汤定一
\item[情况] 比赛后通过
\item[题意]
给AI的石头剪刀布的策略:n条状态,每条状态有该状态出的招和如果对手出了R/P/S分别跳向的下一个状态。你不知道AI初始状态,求你的状态机使得前$10^9$局赢AI。
\item[题解]
每一个状态记录对手的可能状态,刚开始是全部。我出一个至少能赢可能状态中其中一个的招,根据AI的反应将分化向3个新状态。如果某个集合与其祖先的集合相同,则不分化,连一条边至该祖先。
\end{description}

\problem{我是F题的标题}

\begin{description}
\item[负责] 负责一、负责二
\item[情况] 比赛中通过 - 233min(1Y)
\item[题意]
我是题意。
\item[题解]
我是题解。
\end{description}

\problem{我是G题的标题}

\begin{description}
\item[负责] 负责一、负责二
\item[情况] 比赛中通过 - 233min(1Y)
\item[题意]
我是题意。
\item[题解]
我是题解。
\end{description}

\problem{我是H题的标题}

\begin{description}
\item[负责] 负责一、负责二
\item[情况] 比赛中通过 - 233min(1Y)
\item[题意]
我是题意。
\item[题解]
我是题解。
\end{description}

\problem{我是I题的标题}

\begin{description}
\item[负责] 负责一、负责二
\item[情况] 比赛中通过 - 233min(1Y)
\item[题意]
我是题意。
\item[题解]
我是题解。
\end{description}

\problem{Jokewithpermutation}

\begin{description}
\item[负责]马天翼
\item[情况] 比赛中通过 - 37min(1Y)
\item[题意]
给出一个1到n的排列组成的数字,求原先的排列。
\item[题解]
搜。
\end{description}

\problem{Knockout Racing}

\begin{description}
\item[负责] 金杰
\item[情况] 比赛中通过 - 38min(1Y)
\item[题意]
x轴上有n辆车在ai~bi之间以速度1来回移动,m次询问,在t秒l~r之间有多少辆车。
\item[题解]
简单题。\\
nm很小,直接暴力统计。
\end{description}

\section{总结}

我是总结。

\end{document}
