\documentclass[a4paper, 11pt, nofonts, nocap, fancyhdr]{ctexart}

\usepackage[margin=60pt]{geometry}

\setCJKmainfont[BoldFont={方正黑体_GBK}, ItalicFont={方正楷体_GBK}]{方正书宋_GBK}
\setCJKsansfont{方正黑体_GBK}
\setCJKmonofont{方正仿宋_GBK}

\CTEXoptions[today=small]

\pagestyle{plain}

\renewcommand{\thesubsubsection}{Problem \Alph{subsubsection}.}
\newcommand{\problem}[1]{\subsubsection{#1}}

\title{Fudan ACM-ICPC Summer Training Camp 2015}
\author{Team 6 汤定一/马天翼/金杰}
\date{2015年8月10日}

\begin{document}

\maketitle

\section{概况}

本场训练,我们队伍在比赛中完成了7道题目,比赛后完成了2道题目,共完成9道题目。

\section{训练过程}

开局mty写A,1次通过。tdy写宿命之F,文件wa一发后AC。jj写E,wa1次后通过。mty写B,1次通过。mty写D,wa1次后通过。tdy写了C,但错了。此时130min。jj写G,WA。tdy与mty一起写C,修改了tdy错误的算法,算法对了后wa了一发补上了讨论,通过。jj修改错误后G题仍然wa。看板全队去看H与J。mty发现G没有考虑水平垂直情况,jj修改后通过。jj写H,最后交了两发,因为vector<pair<> >中两个>连写CE。赛后提交re。tdy在280min想出了H的网络流做法,但没有时间实现。

\section{解题报告}

\problem{Aztec Pyramid}

\begin{description}
\item[负责] 马天翼
\item[情况] 比赛中通过 - 14min(1Y)
\item[题意]
给n个木块,问最多搭几层,使木块都稳定。稳定的定义是,该木块在地上,或者该木块的下方木块周围都有木块。
\item[题解]
第一层块数1
第二层1+3+1
第三层1+3+5+3+1
以此类推即可。
\end{description}

\problem{Battleship}

\begin{description}
\item[负责] 马天翼
\item[情况] 比赛中通过 - 68min(1Y)
\item[题意]
给一个10*10的矩阵,按数字顺序轰炸该点。要求用一个1*4,2个1*3,3个1*2和4个1*1,放到该矩阵上,相互不相邻,不相交,求一种方案使这些小矩阵被轰炸完时间最迟。
\item[题解]
找到100,覆盖一个1*1,之后随便构造就行了。
\end{description}

\problem{Chemistry}

\begin{description}
\item[负责] 汤定一、马天翼
\item[情况] 比赛中通过 - 220min(2Y)
\item[题意]
求有n个碳原子的烷烃的同分异构体的个数。
\item[题解]
先求烷基的个数,在枚举重心和支链数和支链长进行讨论即可。
\end{description}

\problem{Deepest Station}

\begin{description}
\item[负责] 马天翼
\item[情况] 比赛中通过 - 131min(2Y)
\item[题意]
给两个点,找一个中间点,使两个点到这个点的连线都与水平面呈45度角。
\item[题解]
直接推公式。
\end{description}

\problem{Electricity}

\begin{description}
\item[负责] 金杰
\item[情况] 比赛中通过 - 55min(2Y)
\item[题意]
两种插头。A插板能插到B插座上,B插板插在A上。机器只能插在A上。给所有的插板规格,问最多插多少机器。
\item[题解]
模拟题。A插板能插就插,插不下了多搞个B插板。
\end{description}

\problem{Final Standings}

\begin{description}
\item[负责] 汤定一
\item[情况] 比赛中通过 - 44min(2Y)
\item[题意]
给定n个车手,总积分为p,前k名有d种不同的积分,问可能的方案,若不可能则输出"Wrong information"
\item[题解]
d=1时,先把p/k积分均分给前k个车手,剩下的均分给n-k个车手,注意若n-k个车手的最大积分要小于p/k。若d!=1,只需把积分按1,2,3,...依次给d-1,d-2,d-3,...剩余的积分给第一名即可,注意判断第一名积分是否大于第二名的即可。
\end{description}

\problem{Grid}

\begin{description}
\item[负责] 金杰
\item[情况] 比赛中通过 - 243min(4Y)
\item[题意]
在纸上画一条直线。再画一些等距竖线和等距横线,与该直线相交得一系列点。现在只给所有点坐标,问竖线横线怎么画的。
\item[题解]
按x排序取前3个点。则必然有两个点都是横线上的或都是竖线上的。枚举所有前3个点的情况,得竖(横)线方案,然后根据剩下的点求横(竖)线方案即可。
\end{description}

\problem{Hospital}

\begin{description}
\item[负责] 汤定一、金杰
\item[情况] 比赛后通过
\item[题意]
有一些当班护士和闲置护士。给出每个岗位有哪些护士可以胜任。1.问哪些人可以休息不出乱子。2.问哪两个人本来可以休息但是两个人一起休息就不行。
\item[题解]
写了两个解法:\\
金杰:问题1只要从每个闲置护士开始bfs一遍即可。问题2就枚举哪个人现在休息,把他删掉,重新跑bfs。原先能休息的现在不能休息就加入答案。原先能休息现在不能休息的人之间也加入答案。\\
汤定一:网络流,源点向闲置护士连容量为1的边,每个护士拆成两个点,两个点之间连一条容量为1的边。若a能替代b,从a往b连容量为1的边,护士向汇点的边动态建。第一问的答案即为从源点顺着有流量的边能到达的点。第二问枚举每个能单独休假的当班护士,动态向汇点连一条容量为1的边,从源点向汇点流一次,再bfs从源点不能到达的点但能单独休假的护士即为答案加到答案方案里即可。
\end{description}

\problem{Intelligent Design}

\begin{description}
\item[情况] 尚未通过
\end{description}

\problem{Juggle}

\begin{description}
\item[负责] 马天翼
\item[情况] 比赛后通过
\item[题意]
一列数列(n的全排列)有五个参数:逆序对数、相邻逆序对数、最长上升子序列长度、最长上升子串长度、不动点个数。现给出两个数列的五个参数的大小关系,求构造两个符合要求的长度为l的数列。
\item[题解]
由于五个参数的所有情况种数是$3^5=243$种,经过搜索发现前7个数的全排列就可以找到以上所有情况,因此先预处理前七个的情况,然后暴力枚举找出符合要求的,之后将后l-7个数倒序排放即可。
\end{description}

\problem{Kingdom Subdivision}

\begin{description}
\item[情况] 尚未通过
\end{description}

\problem{Log Analysis}

\begin{description}
\item[情况] 尚未通过
\end{description}

\section{总结}

多交流。

\end{document}


