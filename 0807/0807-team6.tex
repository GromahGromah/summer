\documentclass[a4paper, 11pt, nofonts, nocap, fancyhdr]{ctexart}

\usepackage[margin=60pt]{geometry}

\setCJKmainfont[BoldFont={方正黑体_GBK}, ItalicFont={方正楷体_GBK}]{方正书宋_GBK}
\setCJKsansfont{方正黑体_GBK}
\setCJKmonofont{方正仿宋_GBK}

\CTEXoptions[today=small]

\pagestyle{plain}

\renewcommand{\thesubsubsection}{Problem \Alph{subsubsection}.}
\newcommand{\problem}[1]{\subsubsection{#1}}

\title{Fudan ACM-ICPC Summer Training Camp 2015}
\author{Team 6 汤定一/马天翼/金杰}
\date{2015年8月7日}

\begin{document}

\maketitle

\section{概况}

本场训练,我们队伍在比赛中完成了8道题目,比赛后完成了0道题目,共完成8道题目。

\section{训练过程}

mty写B,交错题后AC。之后tdy写A,1次通过。mty写E,tdy写J,70min相继通过。接着tdy过了H。90min的时候迟到的jj出现了。mty与jj讨论了I,mty的想法是对的,mty去写I。jj将F题意告诉tdy,tdy想出做法,去写F。jj去想D和G。I过了,F发现反例,jj去想F,想出另一种解法,WA一发(忘记乘2)PE一发后通过。看板有人过C,tdy想出了C,写C,WA,jj重写C,重写过程中tdy通过了C。mty对G打表找规律,没找到。

\section{解题报告}

\problem{Decoding Baby Boos}

\begin{description}
\item[负责] 汤定一
\item[情况] 比赛中通过 - 41min(1Y)
\item[题意]
给定字符串,给定m个操作,每次操作把字符串中的一个字符变成另一个,求最后字符。
\item[题解]
先求每个字符最后会变成什么字符,再扫一遍即可。
\end{description}

\problem{And Or}

\begin{description}
\item[负责] 马天翼
\item[情况] 比赛中通过 - 29min(3Y)
\item[题意]
我是题意。
\item[题解]
我是题解。
\end{description}

\problem{A game for kids}

\begin{description}
\item[负责] 汤定一
\item[情况] 比比赛中通过 - 261min(3Y)
\item[题意]
给定n个点的树,每个点上可以取Li到Hi个石子,对每条路径,每种取石子方案的值为路径上每个点所取石子树的最大公约数。问每个值能通过多少种不同的方法得到。从u到v与从v到u为一种。
\item[题解]
树形dp,对一个点统计经过它或它子树的路径即可。因为Li和Hi的范围是50,,所以时间复杂度$n*50^2$
\end{description}

\problem{Flood in Gridland}

\begin{description}
\item[情况] 尚未通过
\end{description}

\problem{Refraction}

\begin{description}
\item[负责] 马天翼
\item[情况] 比赛中通过 - 70min(2Y)
\item[题意]
我是题意。
\item[题解]
我是题解。
\end{description}

\problem{Reverse Polish Notation}

\begin{description}
\item[负责] 金杰
\item[情况] 比赛中通过 - 208min(6Y)
\item[题意]
给一个只有数字a和+的后缀表达式,让使用最少的操作数使得表达式合法。操作为:添加1个a或+,或交换相邻的两个字符。
\item[题解]
先在前面补a或后面补+使得最后a比+多1个。a视为1,+视为-1,求前缀和,要求任何一个位置的值不低于1。因为一次交换操作的作用是把最小值+2,如果有两个最小值不如前面补a后面补+,因此若只有一个最小值就交换,之后前后补。
\end{description}

\problem{Just Some Permutations}

\begin{description}
\item[情况] 尚未通过
\end{description}

\problem{Load Balancing}

\begin{description}
\item[负责] 汤定一
\item[情况] 比赛中通过 - 94min(1Y)
\item[题意]
给定160以内的很多数(可重复),问在划分为四段的情况下,怎样划分才能让每段的个数更接近。
\item[题解]
暴力枚举三个分段点即可。
\end{description}

\problem{Volume of Revolution}

\begin{description}
\item[负责] 马天翼
\item[情况] 比赛中通过 - 135min(1Y)
\item[题意]
我是题意。
\item[题解]
我是题解。
\end{description}

\problem{Maximum Score}

\begin{description}
\item[负责] 汤定一
\item[情况] 比赛中通过 - 72min(2Y)
\item[题意]
给定n个数,每个数出现f次,问怎样排列这些数,使得表达式值最大。
\item[题解]
显然,排成单峰最优,即可计算答案。方案数即为每个相同的数既可以选择在单峰的左边也可以选择右边,逐一乘起来即可。
\end{description}

\section{总结}

写完题马上下,换人上去写。找不出错可以多重写,其实耗时不会很多。尽量保证每个人手上都有东西写。

\end{document}