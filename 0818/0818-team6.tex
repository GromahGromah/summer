\documentclass[a4paper, 11pt, nofonts, nocap, fancyhdr]{ctexart}

\usepackage[margin=60pt]{geometry}

\setCJKmainfont[BoldFont={方正黑体_GBK}, ItalicFont={方正楷体_GBK}]{方正书宋_GBK}
\setCJKsansfont{方正黑体_GBK}
\setCJKmonofont{方正仿宋_GBK}

\CTEXoptions[today=small]

\pagestyle{plain}

\renewcommand{\thesubsubsection}{Problem \Alph{subsubsection}.}
\newcommand{\problem}[1]{\subsubsection{#1}}

\title{Fudan ACM-ICPC Summer Training Camp 2015}
\author{Team 6 汤定一/马天翼/金杰}
\date{2015年8月18日}

\begin{document}

\maketitle

\section{概况}

本场训练,我们队伍在比赛中完成了3道题目,比赛后完成了3道题目,共完成6道题目。

\section{训练过程}

开局mty写03,wa了以后发现没那么简单。jj写05,没过样例,喊tdy马上上来重写。jj去读题,发现04是水题,准备写。tdy因为公式推错wa2发,ij打反wa1发,74min通过。jj觉得好像没写错,把04代码交给mty看,mty也觉得题意和代码都没错,于是放弃。tdy写01,jj和mty搞07。tdy发现公式错了,mty去写07。tdy回来写01,仍然未过,发现是题意读错,于是再换07。期间jj在读题和想03和08。jj莫名想到04中的一个特例,于是修改后通过,此时还有2小时。mty和jj一起看07代码找错,终于改对之后提交,TLE。然后jj想到07一个简单的结论,于是省了一个搜索,但tdy在机子上写01。01在第四个小时通过,还剩1小时。mty修改07之后,wa了,修改了几次仍然wa,因为代码是之前的改的,所以很长而且逻辑混乱,jj决定重写07,此时还有40min。tdy想出了03,在强调肯定没有错而且写的很快之后,jj将机子让给tdy。03没有过样例,于是jj继续重写07,此时还有20min。07没过样例,换03,此时还有8min。07读代码找到错误,在还有4min提交了一次,wa了。剩下4min继续尝试03,至结束未通过。07在赛后调试10min后通过。

\section{解题报告}

\problem{Expression}

\begin{description}
\item[负责] 汤定一
\item[情况] 比赛中通过 - 238min(3Y)
\item[题意]
给定n个数的表达式,每次任选一个运算符,把它两边的数按该运算符运算后放回该位置,直到最后只剩一个数。问所有选取的序列得到的答案和。
回车不换行。
\item[题解]
$n^3$动规。从小到大枚举区间的长度,区间的答案等于以每个运算符为最后一个运算符得到的答案和,注意要乘以组合数。
\end{description}

\problem{Hack it!}

\begin{description}
\item[负责] 汤定一
\item[情况] 赛后通过
\item[题意]
给定左括号的权值p,右括号的权值q,问能否有两个长度相等且小于等于10000的仅包含左右括号的串,它们在base进制下对r取余的值相等。
\item[题解]
构造,用两种串()(),(())构造,即对2500个位置进行构造,()()为w1,(())为w2,若w1>=w2,则记为ci=w1-w2,否则为ci=w2-w1,我们发现2500个位置有2500个ci值,能乘{-1,0,1},每次取最大的两个值,把它们的差插回去,看有没有相等的值,一直做到有相等的为止,再构造答案即可。
\end{description}

\problem{GCD Tree}

\begin{description}
\item[情况] 尚未通过
\end{description}

\problem{Too Simple}

\begin{description}
\item[负责] 金杰
\item[情况] 比赛中通过 - 175min(4Y)
\item[题意]
有m个从集合1-n到集合1-n的映射,有一些映射不知道,表示成-1,问最后有多少方案能使所有的i经过所有映射后等于i。
\item[题解]
如果有多个-1,则其他-1随便排,让最后一个-1排成满足答案的即可。如果没有-1,则直接判断是否满足答案。注意,即使有-1,如果某个映射使集合size变小了也是永远不能满足答案的。
\end{description}

\problem{Arithmetic Sequence}

\begin{description}
\item[负责] 汤定一
\item[情况] 比赛中通过 - 74min(4Y)
\item[题意]
给定n个数的数列。给定d1、d2,若一个区间满足区间由以d1、d2两个等差数列连接而成,则答案+1,问答案为多少。
\item[题解]
当d1!=d2时,以每个点转折点统计答案,当d1=d2时,找到满足条件的区间长度len,答案加上len*(len+1)/2即可。
\end{description}

\problem{Persistent Link/cut Tree}

\begin{description}
\item[负责] 汤定一
\item[情况] 赛后通过
\item[题意]
给定m棵树,每棵树i都是由ai(ai<i)的x结点和bi(bi<i)的y结点以权值li的边连接起来。问每棵树的所有点对的距离和。
\item[题解]
两个树合起来的答案等于它们分别的答案加上以x为根和以y为根的树以li为边权合起来的值。有两种询问,1.以点x为根的树所有点到x点的距离和,因为每颗数都是由两棵树合成的,可以用类线段树的思维计算出。2.一棵树中两个点的距离,若x、y在同一子树中,则继续递归下去,否则为x到ai子树的根的距离加上y到bi子树的根的距离加上ai连接bi的边权,用记忆化搜索即可。
\end{description}

\problem{Travelling Salesman Problem}

\begin{description}
\item[负责] 金杰、马天翼
\item[情况] 比赛后通过
\item[题意]
n*m的棋盘上每个点都有一个非负的数,问从左上角走到右下角最大能收集到的数字和。
\item[题解]
如果n或m是奇数,则走蛇形即可。否则至少有1个点会到不了。如果i+j为奇数,发现一定能构造只牺牲这个点的方案。如果i+j为偶数,发现至少牺牲另一个i+j为奇数的点,所以绝对会经过所有偶数点。于是找奇数点里值最小的即可。
\end{description}

\problem{Goldbach's Conjecture}

\begin{description}
\item[情况] 尚未通过
\end{description}

\problem{Random Inversion Machine}

\begin{description}
\item[情况] 尚未通过
\end{description}

\problem{Sometimes Naive}

\begin{description}
\item[情况] 尚未通过
\end{description}

\section{总结}

01虽然也是一个人啃了很多时间,但是每隔一段时间都有一个重大发现,所以没有发生在一个阶段卡半小时的现象,是没有问题的。\\04看起来花掉的时间不多,其实只有5min的代码,是特殊情况没有考虑到,在刚开局还有很多陌生题的情况下两个人去找有没有问题而花去很多时间才是失误。\\
07的问题很多,首先是在一个人没有出解的情况下就将思路告诉另一个人一起想,所以两个人直接在思考「这种格子怎么走才最省」这样的子问题,其实完全可以避免进入这个问题,所以在没有头绪之前,最好独立思考。\\
然后是代码问题,07因为一改再改,中间还套个搜索,然后又加输出方案什么的,代码相当乱。而且代码重用的很少,复制了好多次,每次修改都要这里改那里改,100行的代码写成了300行。所以宁愿多花一点时间,把代码写的漂亮一点,如果错了找错改错反而更节省时间,出现低级错误的概率也会变小。说到底写代码的时间是真的不长的,能减少找错的时间才是王道。\\
听队友思路时不能嗯嗯嗯的就过去了,没听懂就说出来,否则不如直接写。

\end{document}


