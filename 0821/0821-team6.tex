\documentclass[a4paper, 11pt, nofonts, nocap, fancyhdr]{ctexart}

\usepackage[margin=60pt]{geometry}

\setCJKmainfont[BoldFont={方正黑体_GBK}, ItalicFont={方正楷体_GBK}]{方正书宋_GBK}
\setCJKsansfont{方正黑体_GBK}
\setCJKmonofont{方正仿宋_GBK}

\CTEXoptions[today=small]

\pagestyle{plain}

\renewcommand{\thesubsubsection}{Problem \Alph{subsubsection}.}
\newcommand{\problem}[1]{\subsubsection{#1}}

\title{Fudan ACM-ICPC Summer Training Camp 2015}
\author{Team 6 汤定一/马天翼/金杰}
\date{2015年8月21日}

\begin{document}

\maketitle

\section{概况}

本场训练,我们队伍在比赛中完成了5道题目,比赛后完成了3道题目,共完成8道题目。

\section{训练过程}

开局mty发现H是简单题,跟jj讲过之后马上敲,1A。然后tdy看到了J,发现可做,跟mty和jj讲过之后去敲,RE。在做J的期间jj想出了A,因为要java于是又跟mty讨论出了C,tdy下机让jj去写C。之后mty一直在读题,读完了所有题。tdy下来看了一会儿代码,又RE了一次发现是清空问题,AC。C题wa了以后jj下来看代码,又wa了一发之后发现判断条件写错,又跟mty讨论出了正确的判断条件,AC。然后tdy去写I,mty把读过的题都告诉jj,jj想出了F,可以准备写。mty跟jj讲了D的做法,可行,于是mty去想细节。I题下机了,jj去写F,因为前几天写了好多计算几何,一次过样例一次AC。然后继续写A,A题过样例了,tdy放下手中的题,来把C++翻译成java,jj在一边看着,改完以后提交,wa了。mty找到A的反例,jj去想。然后mty去写D。tdy去想I,发现不可做,放弃去想B。jj修好了A的bug以后发现其他地方还有问题,并且不能解决,于是想D。jj想出了不用判各种局面的方法,去告诉mty,mty正要写麻烦的各种局势判断,代码稍作修改就换成jj的方法,之后jj在旁边看着mty写D。然后D题wa了,tdy去写B,tdy觉得B越搞越麻烦,要写的东西很多。mty和jj在读D的代码,各发现了一些错误之后提交,AC。

\section{解题报告}

\problem{Naughty fairies}

\begin{description}
\item[负责] 金杰
\item[情况] 比赛后通过
\item[题意]
给$n,m<10^{500}$,可以对m做+1/-1/*2三种操作,问最少多少次操作能使m变成n。
\item[题解]
f[i]=从i变n的最少次数。若i为偶数,则f[i/2]=f[i]+1;若i为奇数,则f[i/2]=f[i]+2;注意f[i]可以从f[i*2]和f[i*2+1]转移过来,并且走其他路显然是不合算的。那么每次n/=2,更新f[n/2]和f[n/2+1],然后两个数都/2去处理。然后发现每层都只有2个值要计算,效率log。最后再计算每个计算过的f[i]+|i-m|的最小值。BigInteger。
\end{description}

\problem{Prison Break}

\begin{description}
\item[负责] 马天翼
\item[情况] 比赛后通过
\item[题意]
给出一个矩阵,D不可走,S可走,F是出发点,G是能量池,要走完所有的Y算完成任务,每走一步消耗一个能量,每个能量池可以补充满一次能量,问初始背包容量最小值。
\item[题解]
二分bfs即可,因为能量池和Y的数量不超过15,用一个三维数组记录当前这个位置的能量池和Y的状态即可。
\end{description}

\problem{To Be an Dream Architect}

\begin{description}
\item[负责] 金杰、马天翼
\item[情况] 比赛中通过 - 103min(3Y)
\item[题意]
n阶魔方中删去m条,问共删去多少块。
\item[题解]
先去掉不合条件的和重复的条,那么每块最多被3条消去。m条两两求交点,被提到两次的算重复一次,被提到6次的算重复两次,减去就是答案。
\end{description}

\problem{Gomoku}

\begin{description}
\item[负责] 马天翼
\item[情况] 比赛中通过 - 291min(2Y)
\item[题意]
给一个五子棋局,问三步之内是否有人能肯定获胜。
\item[题解]
记录先手和后手的制胜点有几个,先手若有1个及以上的制胜点那么肯定赢,若后手有两个及以上的,后手赢,若先手无,后手有一个,先手必定会走后手的制胜点,那么走完再计算即可,若先手无后手无,则枚举先手的第一步,做相同操作即可。
\end{description}

\problem{Gunshots}

\begin{description}
\item[情况] 尚未通过
\end{description}

\problem{Rotational Painting}

\begin{description}
\item[负责] 金杰
\item[情况] 比赛中通过 - 155min(1Y)
\item[题意]
给一个多边形,问有多少种形态可以稳定的放在桌面上。
\item[题解]
稳定的条件是重心不落在地面支撑点外。先求出多边形重心。然后对多边形求一次凸包,每一条凸包上的边就是一个可以放在桌面上的形态,但未必稳定,再用点积判断重心是否落在线段内。
\end{description}

\problem{Traffic Real Time Query System}

\begin{description}
\item[情况] 尚未通过
\end{description}

\problem{National Day Parade}

\begin{description}
\item[负责] 马天翼
\item[情况] 比赛中通过 - 15min(1Y)
\item[题意]
有n个人列队,只能左右移动,给出初始坐标,问最少移动几步可以拍成矩形。
\item[题解]
枚举纵坐标即可。
\end{description}

\problem{Searchlights}

\begin{description}
\item[负责] 汤定一
\item[情况] 比赛后通过
\item[题意]
给定n*m的矩形,每个数字为对应位置上的灯的最大强度,一个亮度为k的灯可以使用0~k的亮度,k的亮度可以照亮上下左右距离为k以内的位置。一个位置合法需满足两个条件中的任意一个:1.它有灯且亮度不为0。2.它能被左右的一盏灯照亮且被上下的一盏灯照亮。要求每盏灯的亮度相等,求最小的使每个位置合法的亮度。
\item[题解]
枚举1~10000的亮度。对整个图建立十字链表,为了快速求出仅使用亮度大于k的灯且能覆盖全图的最小的光强。按光强顺序依次删除灯,当枚举到的亮度与最小所需光强相等时即为所求答案。
\end{description}

\problem{Infinite monkey theorem}

\begin{description}
\item[负责] 汤定一
\item[情况] 比赛中通过 - 71min(3Y)
\item[题意]
给定每个小写字母出现的概率,问一个串中出现给定串的概率。
\item[题解]
ac自动机初始化,在ac自动机上动规即可,答案为1-ac自动机上除了最后一个点的概率。
\end{description}

\section{总结}

虽然A题也可以用DP做,但是规律,或者叫性质还是比较显然的,但是没有往那方面想,这个跟昨天是一样的失误。\\
多开了3道题,花了很多时间之后才发现之前的想法是错的而没有想到正解,这个就是没想清楚就去写了。\\
会java的只有一个人,导致做高精要投入两个人的战斗力。

\end{document}

