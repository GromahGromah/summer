\documentclass[a4paper, 11pt, nofonts, nocap, fancyhdr]{ctexart}

\usepackage[margin=60pt]{geometry}

\setCJKmainfont[BoldFont={方正黑体_GBK}, ItalicFont={方正楷体_GBK}]{方正书宋_GBK}
\setCJKsansfont{方正黑体_GBK}
\setCJKmonofont{方正仿宋_GBK}

\CTEXoptions[today=small]

\pagestyle{plain}

\renewcommand{\thesubsubsection}{Problem \Alph{subsubsection}.}
\newcommand{\problem}[1]{\subsubsection{#1}}

\title{Fudan ACM-ICPC Summer Training Camp 2015}
\author{Team 6 汤定一/马天翼/金杰}
\date{2015年8月4日}

\begin{document}

\maketitle

\section{概况}

本场训练,我们队伍在比赛中完成了4道题目,比赛后完成了5道题目,共完成9道题目。

\section{训练过程}

开局tdy写07,过。Mty写05,有问题,tdy与mty调试通过,wa6发,62min过。Tdy写02,没有发现性质,用nlogn的方法写,没过样例,推样例,发现简单解法,89min过。Tdy写10,一小时后没过。Jj将06算法告诉tdy,tdy立刻继续写06,100分钟后没过。先是RE,改手工栈,数组越界,检查代码发现不用手工栈,是dfs没return,修改后TLE。优化代码,仍然TLE。jj重写,过程中发现TLE的原因,加前向弧优化,WA。此时269min。tdy将10改为费用流,289min过10。此时发现06WA在理解错题意,mty和jj口述代码,但变量命名习惯不同,对tdy的代码使用了错误的变量,还是wa。赛后修改变量名通过。

\section{解题报告}

\problem{MZL's Circle Zhou}

\begin{description}
\item[负责] 金杰、汤定一
\item[情况] 比赛后通过
\item[题意]
从两个串中各取一个子串拼起来,问一共有多少不同的结果。
\item[题解]
如果有两段重合,取中间任何一个点分隔都是一个方案,但这些方案的结果相同,因此只考虑统计以最后一个点为分隔。对26个字母c,统计cntA[c]为A中以c为结尾的不同字符串的个数,cntB[c]为B中以c为开头的不同字符串的个数。对于每个c,(A中不同的子串数-cntA[c])*cntB[c]就是以c为结尾分隔符,并且A不能向后再扩展一位的结果。再加上AB中子串数就是答案。其中cntA与cntB可以由后缀自动机得出。注意答案大于longlong。
\end{description}

\problem{MZL's xor}

\begin{description}
\item[负责] 汤定一
\item[情况] 比赛中通过 - 89min(1Y)
\item[题意]
给定序列A,求2*a1xor2*a2xor...xor2*an
\item[题解]
按照题意做即可。
\end{description}

\problem{MZL's combat}

\begin{description}
\item[情况] 尚未通过
\end{description}

\problem{MZL's game}

\begin{description}
\item[负责] 汤定一
\item[情况] 赛后通过
\item[题意]
有n男生在进行游戏,每一次随机选择一个男生出局,他会攻击仍在游戏中的所有人,每个人在被攻击时有p的概率出局,1-p的概率继续游戏。问每个人被攻击0到n-1次的概率。
\item[题解]
题意等价于每次随机选择一个男生,若他已经出局,则什么也不做,若他还在游戏中则让他攻击其他人。动规,记f[i][j]为前i轮选到了j个在游戏中的男生的概率。$f[i][j]=f[i-1][j]*(1-((1-p)^j))+f[i-1][j-1]*((1-p)^(j-1))$,最后被攻击k次的答案为0~$n-1f[i][k]*((1-p)^k)$。
\end{description}

\problem{MZL's chemistry}

\begin{description}
\item[负责] 马天翼
\item[情况] 比赛中通过 - 62min(7Y)
\item[题意]
我是题意。
\item[题解]
我是题解。
\end{description}

\problem{MZL's endless loop}

\begin{description}
\item[负责] 汤定一、金杰
\item[情况] 比赛后通过
\item[题意]
给个无向图,请将每条边都定向成有向边,使得每个点的入度出度相差<1
\item[题解]
有环转一圈,完了是森林。每棵树dfs求解即可。找环就dfs,找到一条返祖边就回溯删边。\\
一个没有实现过的简单做法:度为奇数的点数量是偶数。随意配对连边,跑一遍欧拉路即是答案。
\end{description}

\problem{MZL's simple problem}

\begin{description}
\item[负责] 汤定一、马天翼
\item[情况] 比赛中通过 - 16min(1Y)
\item[题意]
给定三种操作,加入一个数,删掉最小的数,询问最大的数。
\item[题解]
用set模拟即可。
\end{description}

\problem{MZL's munhaff function}

\begin{description}
\item[负责] 汤定一
\item[情况] 赛后通过
\item[题意]
给定一个序列A,f[i][j]=f[i][j>>1]+s[i],f[i][j]=f[i-1][j+1],其中s[i]=a[i]+a[i+1]+……+a[n]。求最小的f[n][1]。
\item[题解]
由题目名字知此题可以用huffman求解,仔细分析可知确实能用huffman。
\end{description}

\problem{MZL's Border}

\begin{description}
\item[负责] 马天翼、汤定一
\item[情况] 比赛后通过
\item[题意]
给出一个斐波那契字符串,问第n个串前m位的前max位与后max位相等。
\item[题解]
打表找规律,用java写。
\end{description}

\problem{MZL's City}

\begin{description}
\item[负责] 汤定一
\item[情况] 比赛中通过 - 233min(1Y)
\item[题意]
给定n个城市,三种操作:1.修好x到y之间的路2.有很多路断掉了3.修好从x出发能到达的城市。地三种操作每次最多能修好k座城市,问最多能修好多少座城市,并输出字典序最小的方案。
\item[题解]
费用流,最小费用最大流。源点到第三种操作连流量为k的边,它的费用一会讨论。每次操作向它能修好的城市连边流量为1费用为0的边,每座城市向汇点连流量为一费用为0的边。源点流向第三种操作的费用逐次操作递减,此举是为了满足字典序最小。因为dinic网络流不能保证每次都是从后往前更新的。
\end{description}

\section{总结}

变量名要有意义,某些常用变量要有队内共识,代码是有可能交给队友看的。如果一个人身上的题比较多,尽可能的分担给队友,否则一卡就完了。

\end{document}