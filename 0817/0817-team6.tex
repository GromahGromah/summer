\documentclass[a4paper, 11pt, nofonts, nocap, fancyhdr]{ctexart}

\usepackage[margin=60pt]{geometry}

\setCJKmainfont[BoldFont={方正黑体_GBK}, ItalicFont={方正楷体_GBK}]{方正书宋_GBK}
\setCJKsansfont{方正黑体_GBK}
\setCJKmonofont{方正仿宋_GBK}

\CTEXoptions[today=small]

\pagestyle{plain}

\renewcommand{\thesubsubsection}{Problem \Alph{subsubsection}.}
\newcommand{\problem}[1]{\subsubsection{#1}}

\title{Fudan ACM-ICPC Summer Training Camp 2015}
\author{Team 6 汤定一/马天翼/金杰}
\date{2015年8月17日}

\begin{document}

\maketitle

\section{概况}

本场训练,我们队伍在比赛中完成了7道题目,比赛后完成了3道题目,共完成10道题目。

\section{训练过程}

开局mty写J,一次通过。jj写H,wa了。tdy看了H的题目,坚持认为jj读错题,而jj认为没有理解错,在jj将信将疑下,写了,通过。mty和tdy讨论后,mty去写E。期间jj和tdy想题,攒下了许多题。90min,E题wa了,tdy马上上去写F。没过样例,调试10min,换jj去写G。tdy在机下发现错误,修改后1次通过,然后jj的G1次通过,此时过去两个半小时。mty一直在做E。tdy和jj都认为在B在残量网络上继续跑就可以,于是tdy去写B,然而TLE。tdy想出了I的做法,将解法告诉jj,jj写I,一次通过,此时3小时22分。因为E消耗了太多时间,mty放弃E,去找B的模板。tdy在mty的帮助下来重写E,通过。mty找到了B的模板觉得应该是的,交给jj,去帮忙搞E了。jj抄完模板,通过。

\section{解题报告}

\problem{Knight's Problem}

\begin{description}
\item[负责] 马天翼
\item[情况] 比赛后通过
\item[题意]
给出一个马可以走的步,问最少几步可以从初始点走到目标点。
\item[题解]
搜索,保证当前马在起点到终点半径为20的路径上即可。
\end{description}

\problem{Nubulsa Expo}

\begin{description}
\item[负责] 金杰、马天翼、汤定一
\item[情况] 比赛中通过 - 280min(2Y)
\item[题意]
给无向图,从中找一个点当汇点,问最小的从1到汇点的最大流。
\item[题解]
最大流即最小割,1一定在割的一边,所以即是求无向图全图最小割。模板题。
\end{description}

\problem{Shade of Hallelujah Mountain}

\begin{description}
\item[负责] 金杰
\item[情况] 比赛后通过
\item[题意]
空间中有一个多面体,给一个点光源和一个平面,问多面体的影子的面积。
\item[题解]
平面是ax+by+cz=d,先将所有点的坐标x平移d/a,这样平面就变成了ax+by+cz=0,一定过原点。\\
然后将法向量(a,b,c)转到z轴上,所有点也跟着转,这样平面就变成xOy面了。再求出每个点在xOy上的投影,跑一遍凸包,求个面积即可。\\
旋转先转z轴,再转x轴,都是逆时针转k,$cos(k)=y/{x^2+y^2}$, x'=x*cos-y*sin, y'=x*sin+y*cos
\end{description}

\problem{Math teacher's homework}

\begin{description}
\item[负责] 汤定一
\item[情况] 赛后通过
\item[题意]
给定n个数的取值范围,问这n个数异或起来结果为k的方案数。
\item[题解]
动规。从二进制低位到高位。对一个二进制位i,如果有第j个数的第i位为1,它可以变成0,则在其他数可以随意变换。为了保证第i位异或之后与k的第i位相等。f[i][j]表示前i个数有j个1的方案数。为了保证不重复,枚举第j个数为第一个在第i位1变0的数字。
\end{description}

\problem{Fermat Point in Quadrangle}

\begin{description}
\item[负责] 汤定一、马天翼
\item[情况] 比赛中通过 - 262min(8Y)
\item[题意]
求四边形的费马点。
\item[题解]
如果有重点,费马点为重点。若为凸四边形,对角线交点为费马点。若为凹四边形,则凹进去的那个点为费马点。凹凸可以用凸包求。
\end{description}

\problem{Computer Virus on Planet Pandora}

\begin{description}
\item[负责] 汤定一
\item[情况] 比赛中通过 - 139min(1Y)
\item[题意]
给定n个子串,一个主串,问哪些子串在主串中出现过。
\item[题解]
子串建AC自动机,主串在AC自动机上跑,AC自动机上的每个点维护一个bitset记录这个点是哪些子串的后缀。
\end{description}

\problem{Farm Game}

\begin{description}
\item[负责] 金杰
\item[情况] 比赛中通过 - 144min(1Y)
\item[题意]
给n种作物的价格和数量。再给很多1个a能换k个b的转换关系。问最多能卖多少钱。
\item[题解]
因为保证无环,是拓扑图,倒着推,把价格更新过去就好了。最后统计一遍。
\end{description}

\problem{Selecting courses}

\begin{description}
\item[负责] 金杰
\item[情况] 比赛中通过 - 88min(2Y)
\item[题意]
给n门课的开放时间,学生每选一节课,下一次选课时间是5分钟后,其他时间无效,问最多多少门课。
\item[题解]
枚举第一门课选课时间,然后模拟跑一遍,每次选结束时间最早的那个即可。
\end{description}

\problem{Let the light guide us}

\begin{description}
\item[负责] 金杰、汤定一
\item[情况] 比赛中通过 - 202min(1Y)
\item[题意]
n*m的棋盘上每行要建一个塔。给每个格点的建塔费用。再给每个格点的魔法值,相邻行的两个塔的距离不能大于两点魔法值之和。求最小费用。
\item[题解]
dp[i][j]表示前i-1行都建了塔,第i行建在j上的最小总费用。每行建一棵线段树,把dp[i][j]更新到这个点左右魔法值的范围内。然后下一行每个点的费用值就是该点左右魔法值范围内的最小值+该点费用。
\end{description}

\problem{A hard Aoshu Problem}

\begin{description}
\item[负责] 马天翼
\item[情况] 比赛中通过 - 28min(1Y)
\item[题意]
给出一个等式,不同字母代表不同数字,现在往前两个数中间填入运算符,以及给字母安排数字,求使等式成立的不同方案数。
\item[题解]
搜索即可。
\end{description}

\section{总结}

写F时只用了10min,但写完没过样例之后没有马上读代码,而是对着样例调试,花了10min最后还是在机下看出来的,在后面攒着3道题并wa着1道题的时候用10min机时去做这个事情很不划算,无论如何写完没过就应该马上读代码。\\
虽然题目有交流,但是写代码这个事情有点各自为战,E题卡住的时候其他人给予的帮助并不够,而因为板上可写题都想出来了(或者说感觉是正解),所以还是同一个人一直啃一道题,这样很不好,要么帮忙读代码,或者直接重写。因为其他人手上也有题所以总想着先写手上的题,可以把题解告诉队友然后换题写,你看tdy想出I交给jj来平均手上的题,这个策略就是不错的,应该推广到重写换题上。

\end{document}