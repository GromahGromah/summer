\documentclass[a4paper, 11pt, nofonts, nocap, fancyhdr]{ctexart}

\usepackage[margin=60pt]{geometry}

\setCJKmainfont[BoldFont={方正黑体_GBK}, ItalicFont={方正楷体_GBK}]{方正书宋_GBK}
\setCJKsansfont{方正黑体_GBK}
\setCJKmonofont{方正仿宋_GBK}

\CTEXoptions[today=small]

\pagestyle{plain}

\renewcommand{\thesubsubsection}{Problem \Alph{subsubsection}.}
\newcommand{\problem}[1]{\subsubsection{#1}}

\title{Fudan ACM-ICPC Summer Training Camp 2015}
\author{Team 6 汤定一/马天翼/金杰}
\date{2015年8月13日}

\begin{document}

\maketitle

\section{概况}

本场训练,我们队伍在比赛中完成了6道题目,比赛后完成了5道题目,共完成11道题目。

\section{训练过程}

开局mty写08,wa3发小错误后AC。mty写07,1次通过。tdy写10,1次通过。jj写06,1次通过。tdy写05,1次通过。此时150min。之后jj说会02三个log的预处理,tdy说之后能用莫队搞,于是开始敲,敲的过程中想出一个log预处理,于是继续写,因为longlong wa两发后ac。此时还剩一小时,jj听到隔壁在说网络流,马上跟mty一起找有什么题可以网络流,发现只有04可以。因为直到4小时都几乎没人过04,于是没有人读04。于是mty读04,把题意告诉jj,发现是简单题,于是jj去敲。tdy搞完02,听了04,觉得可行。之后jj写完,死循环,tdy马上重写。tdy写完费用流部分,决定复制jj的建边过程。jj将代码复制过来,三个人一起调整后终于把两份代码拼到一起。此时还有10min。jj发现建边有问题,多次修改。因为是两个人的代码,找错的时候十分混乱,许多接口都对不上。最后在赛后十分钟才通过。

\section{解题报告}

\problem{Travel with candy}

\begin{description}
\item[负责] 金杰、汤定一
\item[情况] 赛后通过
\item[题意]
在一维上给定n座城市的坐标,从小到大。每座城市有糖果的买卖价格。每走一单位距离就会消耗一单位糖果,背包能装c单位糖果,问走到第n座城市最少需要花多少钱,可能为负数,此时为赚钱。
回车不换行。
\item[题解]
贪心。每到一座城市即装满背包,若背包里一部分糖果的卖价高于买价,即可卖回去。
\end{description}

\problem{The sum of gcd}

\begin{description}
\item[负责] 汤定一、金杰
\item[情况] 比赛中通过 - 250min(3Y)
\item[题意]
给定数列A,Q次询问。每次询问l,r所有子区间的gcd。
\item[题解]
莫队。以每个位置开始和结尾的gcd值最多logn个不同的值,预处理出每一段的端点即可。前缀断点从右往左做,后缀断点从左往右做。在莫队里移动左右端点时每次logn询问即可。
\end{description}

\problem{GCD?LCM!}

\begin{description}
\item[负责] 金杰
\item[情况] 比赛后通过
\item[题意]
$$F(n)=\sum_{i=1}^n\sum_{j=1}^n~[~lcm(i,j)+gcd(i,j)\geq n~] $$\\
$$S(n)=\sum_{i=1}^nF(i)$$\\
给n,输出S[n]
\item[题解]
d=gcd(i,n),a=i/d,b=n/d,lcm(i,n)+gcd(i,n)=abd+d>=n\\
由F[n]推F[n+1]时,由于n固定,枚举d,则b固定,a可O(1)求。但刚好==n的到F[n+1]就失效了,因此要求出T[n]表示刚好等于的,这个求法就是上文说的。那么F[n]就没必要用类似T的方法求,因为lcm(i,n)>=n,所以F[n+1]=F[n]+(2n-1)-T[n];
\end{description}

\problem{Yu-Gi-Oh!}

\begin{description}
\item[负责] 金杰、汤定一
\item[情况] 赛后通过
\item[题意]
游戏王,场面上n只小怪,分AB两种,有m种大怪可以融合,融合条件是一只A一只B融合成一只大怪,A的星+B星=打怪的星。某些大怪有必须谁来融合的要求。问攻击力之和的最大值。
\item[题解]
网络流,A左边一列,B右边一列,两两连边如果可以融合,费用为增加的攻击力。跑最大费用可行流即可。
\end{description}

\problem{Danganronpa}

\begin{description}
\item[负责] 汤定一
\item[情况] 比赛中通过 - 154min(1Y)
\item[题意]
 给定n个主串,m个子串。问每个主串包含子串的总次数,一个子串在主串的不同位置出现算多次。
\item[题解]
AC自动机。以子串建AC自动机,主串在AC自动机上跑并统计答案即可。
\end{description}

\problem{The path}

\begin{description}
\item[负责] 金杰
\item[情况] 比赛中通过 - 112min(1Y)
\item[题意]
给有向图,要求给图中的边标上n以内的长度,使得d[1]<....<d[x]>...>d[n]
\item[题解]
事先标上d[x]的值,从两边往中间标号,从1到n标。\\
bfs,每次取最左边或最右边还没标上距离的,标上++cnt,然后从到他的边中随便找一条,计算两点差当长度,其他边长度定为最大值。因为保证有解,所以每次都一定能拿到最左边或最右边的,否则就不满足拓扑了。
\end{description}

\problem{Cover}

\begin{description}
\item[负责] 马天翼
\item[情况] 比赛中通过 - 81min(1Y)
\item[题意]
给出一个初始矩阵,以及一些操作,把某一行或者某一列都变成某一个数字,再给出目标矩阵,求一种操作的排列,使初始变成目标。
\item[题解]
因为肯定有解,倒着扫一遍即可。
\end{description}

\problem{Clock}

\begin{description}
\item[负责] 马天翼
\item[情况] 比赛中通过 - 38min(4Y)
\item[题意]
给出一个时间,求该时间点三根针之间的角度。
\item[题解]
直接做就行了,注意12-23点是0到11点。
\end{description}

\problem{Geometer's Sketchpad}

\begin{description}
\item[负责] 金杰
\item[情况] 赛后通过
\item[题意]
空间中一些点,有平移,缩放,绕任意轴旋转操作。询问一个点当前坐标和询问一段区间中相邻点距离和。
\item[题解]
将点x,y,z定义成矩阵(x,y,z,1),将三个操作分别求出4*4矩阵,于是就可以用线段树维护第一问。注意到平移旋转不会改变区间内相邻点距离,缩放只同时改变k倍,所以再搞一个线段树支持区间修改,然后修改边缘的两个点即可。
\end{description}

\problem{Zero Escape}

\begin{description}
\item[负责] 汤定一
\item[情况] 比赛中通过 - 62min(1Y)
\item[题意]
给定n个人的值,两个门A、B的值,1~9。两个数的数字根为两数和的各位数字之和,为1~9。一群人进入一扇门的条件为他们的和的数字根为门上的值。问使得每个人进入任意一扇门的方案数。
\item[题解]
背包出数字根为1~9的方案数。最后要注意所有人进入一扇门的讨论。
\end{description}

\problem{tree}

\begin{description}
\item[负责] 汤定一
\item[情况] 赛后通过
\item[题意]
给定一个n个节点的树,每个节点上有一个值。m次询问,两种操作:1.修改一个点的值2.询问一个点与根到它的路径上一个点异或的最大值。
\item[题解]
用dfs序建线段树,每次更改操作影响连续的一段,询问操作询问线段树根到它的最大值。线段树上的每个节点存一个字典树。答案在字典树上找。但是这样做会爆空间。把线段树拆成logn层一层一层地做,每层做一遍询问,更新答案即可。
\end{description}

\section{总结}

永远不要把两个人的代码拼到一起,很乱的。\\
有可能有人会带歪板,尤其是有很多高中生在。在赛程过半之前一定要保证所有题目都读过。


\end{document}


